\section{Introduction}

We consider the simplest possible language.

\begin{figure}[h]
  \begin{tabularx}{\columnwidth}{rlR}
    $t ::=$ & & type \\
            & $t \to t$       & arrow \\
            & $()$            & unit \\
      \\[1ex]
    $e ::=$ & & expression \\
            & $e \ e$         & application \\
            & $\lambda x.e $  & abstraction \\
            & $x$             & variable \\
            & $()$            & unit \\
  \end{tabularx}
  \caption{DumbML, the degenerated language we're considering}
  \label{fig:syntax}
\end{figure}

This leads to the following OCaml definitions:

\begin{ocamlcode}
type typ =
  | TArrow of typ * typ
  | TUnit

type expr =
  | ELambda of string * expr
  | EApp of expr * expr
  | EUnit
  | EVar of string
\end{ocamlcode}

Not only do we take an excessively simple language, but we also take an
excessively stupid algorithm which, instead of using a notion of polymorphism,
rather, tries several solutions. That is, when type-checking $x$, instead of
using a unification variable and generalizing (ML) or considering one possible
type (simply-typed lambda-calculus), we are going to try several possible types
for $x$, one after another, and see which of these types ``work''.

The reason why are taking such a surprising approach is that, as we mentioned
earlier, in Mezzo, we cannot take a classic, unification-based approach. Mezzo
is at the frontier between a program logic and a type system; therefore, the
usual approaches that we use when writing a type-checker no longer work. Namely,
we had to compromise on the following:
\begin{itemize}
  \item the type-checker needs to backtrack; therefore, we deal with a stream
    (lazy list) of solution for each expression rather than one possible type,
    or zero type if the expression failed to be type-checked;
  \item error reporting also becomes harder; 
\end{itemize}<++>
