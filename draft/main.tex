\documentclass[nonatbib]{sigplanconf}

\usepackage{hyperref}
\usepackage[T1]{fontenc}
\usepackage[utf8x]{inputenc}
\usepackage{minted}
\usepackage{xspace}
\usepackage{microtype}
\usepackage{paralist}


\newcommand{\code}[1]{\texttt{#1}}
\newcommand{\sref}[1]{\S\ref{#1}}
\newminted{ocaml}{fontsize=\footnotesize}
\newminted{haskell}{fontsize=\footnotesize}

\begin{document}

% For ACM, explicitly specify A4 paper.
% For us, no need to specify it (but check your TeX configuration if
% you insist on producing a specific size).
\special{papersize=8.5in,11in}
\setlength{\pdfpageheight}{\paperheight}
\setlength{\pdfpagewidth}{\paperwidth}

\exclusivelicense

\conferenceinfo{ICFP~'14}{September 1--3, 2014, Copenhagen, Denmark}
\copyrightyear{2014}
\copyrightdata{978-1-nnnn-nnnn-n/yy/mm}
\doi{nnnnnnn.nnnnnnn}

\title{The dumbest type-checker ever}

\authorinfo{
  Pierre-Évariste Dagand \quad
  Jonathan Protzenko
}{
  INRIA
}{http://gallium.inria.fr/blog/}
% \authorinfo{Pierre-Évariste Dagand}
%            {INRIA}
%            {pierre-evariste.dagand@inria.fr}
% \authorinfo{Jonathan Protzenko}
%            {INRIA}
%            {jonathan.protzenko@ens-lyon.org}

\maketitle

\begin{abstract}
  In spite of recent advances in program certification, testing remains a
  widely-used component of the software development cycle. Various flavors of
  testing exist: popular ones include \emph{unit testing}, which consists in
  manually crafting test cases for specific parts of the code base, as well as
  \emph{quickcheck-style} testing, where instances of a type are automatically
  generated to serve as test inputs.

  These classical methods of testing can be thought of as \emph{internal}
  testing: the test routines access the internal representation of the data
  structures to be checked. We propose a new method of \emph{external}
  testing where the library only deals with a \emph{module interface}. The data
  structures are exported as \emph{abstract types}; the testing framework
  behaves like regular client code and combines functions exported by the
  module to build new elements of the various abstract types. Any counter-examples
  are then presented to the user.
\end{abstract}

\category{D.2.5}{Software Engineering}{Testing and Debugging}

\keywords{
functional programming,
testing,
quickcheck
}

% \section{Introduction}

We consider the simplest possible language.

\begin{figure}[h]
  \begin{tabularx}{\columnwidth}{rlR}
    $t ::=$ & & type \\
            & $t \to t$       & arrow \\
            & $()$            & unit \\
      \\[1ex]
    $e ::=$ & & expression \\
            & $e \ e$         & application \\
            & $\lambda x.e $  & abstraction \\
            & $x$             & variable \\
            & $()$            & unit \\
  \end{tabularx}
  \caption{DumbML, the degenerated language we're considering}
  \label{fig:syntax}
\end{figure}

This leads to the following OCaml definitions:

\begin{ocamlcode}
type typ =
  | TArrow of typ * typ
  | TUnit

type expr =
  | ELambda of string * expr
  | EApp of expr * expr
  | EUnit
  | EVar of string
\end{ocamlcode}

Not only do we take an excessively simple language, but we also take an
excessively stupid algorithm which, instead of using a notion of polymorphism,
rather, tries several solutions. That is, when type-checking $x$, instead of
using a unification variable and generalizing (ML) or considering one possible
type (simply-typed lambda-calculus), we are going to try several possible types
for $x$, one after another, and see which of these types ``work''.

The reason why are taking such a surprising approach is that, as we mentioned
earlier, in Mezzo, we cannot take a classic, unification-based approach. Mezzo
is at the frontier between a program logic and a type system; therefore, the
usual approaches that we use when writing a type-checker no longer work. Namely,
we had to compromise on the following:
\begin{itemize}
  \item the type-checker needs to backtrack; therefore, we deal with a stream
    (lazy list) of solution for each expression rather than one possible type,
    or zero type if the expression failed to be type-checked;
  \item error reporting also becomes harder; right now, the derivation library
    we built constructs derivations or presents the caller with an explanation
    for the failure.
\end{itemize}

% \input{anexample}
% \input{representing}
% \input{properties}
% \input{examples}
% \input{relatedwork}
% \input{conclusion}

\bibliographystyle{plain}
\hbadness=10000
\bibliography{local}

\end{document}
